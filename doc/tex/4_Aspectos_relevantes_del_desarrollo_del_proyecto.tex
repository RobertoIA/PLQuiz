\capitulo{4}{Aspectos relevantes del desarrollo del proyecto}

\section{Desarrollo de prototipo}
En el intervalo de tiempo situado entre la selección del proyecto y el comienzo formal de la primera iteración de desarrollo, se dedicaron recursos al desarrollo de un prototipo de la aplicación.
Los objetivos de esta fase eran la construcción de una interfaz gráfica que aproximase el diseño final (aunque no todas las funciones estuvieran implementadas), y el desarrollo de una funcionalidad parcial de la aplicación.
En este caso la funcionalidad parcial era la resolución del primer tipo de cuestiones.

\imagenflotante{prototipo_interfaz}{Interfaz gráfica del prototipo de la aplicación}

El prototipo está limitado a trabajar con un único problema a la vez (ver figura~\ref{fig:prototipo_interfaz}), las opciones de generación no son funcionales y el botón `Generar' produce siempre la misma expresión.
A pesar de estas limitaciones, el prototipo aproxima con bastante precisión la interfaz final, y forma la base de resolución de problemas de manera eficaz.
Es capaz, en concreto, de procesar una expresión regular cualquiera proporcionada como texto, construir a partir de ella un árbol sintáctico, aplicarle el algoritmo de Aho-Sethi-Ullman y mostrar los resultados obtenidos.

\section{Metodología ágil --- Scrum}
El desarrollo del proyecto ha sido enmarcado dentro de una metodología ágil, Scrum\footnote{\url{https://www.scrum.org/}}.
Sin embargo, y debido a las diferentes condiciones entre equipo de desarrollo y las condiciones entre este proyecto y uno más habitual, se han realizado ciertas modificaciones.

En cuanto a roles, volcamos los tres roles que define Scrum («\emph{Product Owner}», «\emph{Development Team}» y «\emph{Scrum Master}») sobre el desarrollador, dado que no tenemos un equipo como tal.
El rol de «\emph{Product Owner}», especialmente la definición de tareas para el \emph{backlog}, se realizará conjuntamente con el tutor.

Se definen \emph{sprints} de dos semanas, entre los cuales se realiza una reunión con el tutor del proyecto, y al final de cada cuál se completa un entregable.
Esta reunión cumple las funciones de «\emph{Sprint Planning Meeting}» y «\emph{Sprint Review}», y en ella se revisan los resultados del \emph{sprint} que termina y se definen los objetivos del que comienza.
No se realizan reuniones diarias («\emph{Daily Scrum}»), resultando innecesarias al contar con un equipo de desarrollo de una persona.

El seguimiento de los \emph{sprint} se realiza mediante el uso de una herramienta online, Pivotal Tracker\footnote{\url{https://www.pivotaltracker.com/s/projects/1026880}}, de manera que resulta accesible a tutor y desarrollador.
La integración de esta herramienta con el sistema de control de versiones utilizado, Github\footnote{\url{https://github.com/RobertoIA/PLQuiz}}, permite la terminación de tareas directamente desde el \emph{commit} que las finaliza.
Se consideran validadas las tareas cuando se completa su documentación (en forma de \emph{javadoc}), se escriben sus pruebas y éstas se ejecutan correctamente.

En el apéndice~\ref{sec:Cappendix} describimos en detalle el desarrollo de cada \emph{sprint}, así como su duración, objetivos y resultados.

\section{Estructura de paquetes}
La filosofía seguida a la hora de diseñar la estructura de la aplicación se centra en los conceptos de modularidad y reusabilidad.
Cada paquete pretende ser independiente, y las dependencias externas se reducen lo más posible.
Esto se traduce en un uso frecuente del patrón `fachada'.

Cada paquete principal tiene una estructura de dos niveles: el primer nivel contiene las clases `fachada', visibles al resto de la aplicación, mientras que el segundo (\ruta{.datos}) contiene la lógica y estructuras de datos internas, utilizadas solo por el paquete `padre'.

Podemos dividir los paquetes principales de la aplicación en tres clases:
\begin{itemize}
	\item Paquete de interfaz gráfica (\ruta{es.ubu.inf.tfg.ui}).
	Contiene las clases que definen la apariencia y comportamiento de la interfaz de usuario.
	Es, por pura necesidad, el paquete con más dependencias externas, dependiendo tanto del paquete que define la representación de documentos, como de los paquetes que definen cada clase de problemas.
	\item Paquete de construcción de documentos (\ruta{es.ubu.inf.tfg.doc}).
	Dependen, por necesidad, de los paquetes que definen las clases de problemas que van a representar.
	\item Paquetes de problemas, como pueden ser \ruta{es.ubu.inf.tfg.regex.thompson} y \ruta{es.ubu.inf.tfg.regex.asu}.
	Contienen las clases encargadas de representar las estructuras internas del problema, sus soluciones, y aquellas necesarias para generarlos.
	Dependen únicamente del paquete de representación de expresiones regulares, y son independientes el uno del otro.
	\item Paquete de representación y procesamiento de expresiones regulares, en concreto \ruta{es.ubu.inf.tfg.regex.parser} y \ruta{es.ubu.inf.tfg.regex.datos}.
	Contienen las clases necesarias para procesar expresiones regulares a partir de cadenas de caracteres, representarlas y operar sobre ellas.
	Son la base sobre la cual trabajan los problemas.
\end{itemize}

\imagen{diagrama_paquetes}{Diagrama de paquetes y sus dependencias}

Como podemos ver, los paquetes encargados del procesamiento de expresiones regulares son la base del problema.
Junto con los paquetes de problemas forman el núcleo de la aplicación, y pueden ser utilizados de manera independiente, al carecer de dependencias externas.

Sobre estos paquetes centrales se aposentan los paquetes de representación de documentos, que toman los datos de los documentos y les dan un formato adecuado.
En un nivel superior está la interfaz de usuario, que toma el conjunto de datos obtenido del resto de paquetes y lo presenta al usuario.

Esta estructura en niveles significa que podemos tomar cualquier paquete, y utilizarlo de manera completamente independiente, siempre que dispongamos de aquellos que se encuentran en niveles inferiores.

\section{Java 8 - \emph{expresiones lambda} y \emph{streams}}
Dos de las novedades introducidas con la versión 8 de Java son los \emph{streams}\footnote{\url{http://docs.oracle.com/javase/8/docs/api/java/util/stream/package-summary.html}} y las \emph{expresiones lambda}\footnote{\url{http://docs.oracle.com/javase/tutorial/java/javaOO/lambdaexpressions.html}}.

Los \emph{streams} permiten manipular colecciones, aplicándoles operaciones típicas de programación funcional, como pueden ser \emph{map} o \emph{reduce}.
Una de las ventajas de estas operaciones son que permiten la aplicación de operaciones a cada elemento de una colección sin iterar explícitamente, o la obtención de subconjuntos de la misma que cumplan una condición cualquiera.

La aplicación de operaciones se realiza mediante \emph{expresiones lambda}.
Las \emph{expresiones lambda} permiten definir una clase de un solo método de manera compacta, encapsulado la operación.

Dado que la salida de la nueva versión de Java ocurrió cuando el desarrollo de la aplicación ya estaba avanzado, la utilización de las nuevas herramientas se da de manera ocasional, y tiene un impacto reducido.

El principal uso de los \emph{stream} se encuentra en las clases encargadas de la generación de expresiones regulares.
Por ejemplo, las operaciones sobre `poblaciones' de expresiones regulares dentro de un algoritmo genético resultan simples si disponemos de medios de `filtrar' subconjuntos dentro de la colección.
Usando la función de evaluación del algoritmo genético como argumento para la operación de filtrado sobre el \emph{stream} permite obtener el conjunto de la población que queremos preservar, mutar o cruzar de manera simple y sin recurrir a la iteración explicita sobre la colección.

\section{Interfaz de usuario}
La interfaz de usuario en la aplicación se divide en tres niveles:
\begin{itemize}
	\item La propia interfaz de la aplicación, desarrollada en Swing, y que es el medio principal de interacción
	\item El panel de vista previa, dentro de la interfaz.
	Muestra la resolución de los problemas añadidos, utilizando HTML y CSS, con los cuales Swing es directamente compatible.
	La generación de imágenes de imágenes se realiza mediante JGraphX, que permite su inserción.
	\item Los documentos exportados, tanto en formato \LaTeX{} como en XML.
	El exportado se realiza mediante el sistema de plantillas de la aplicación, y las imágenes se generan de tres maneras posibles:
	\begin{itemize}
		\item Generadas mediante JGraphX y almacenadas como fichero, en el exportado de documentos \LaTeX{}
		\item Como ficheros \emph{dot} compatibles con Graphviz, en el exportado de documentos \LaTeX{} con Graphviz
		\item Como cadenas de caracteres codificadas en 64 bits, en el exportado de documentos XML compatibles con Moodle
	\end{itemize}
\end{itemize}

\subsection{Sistema de plantillas}
La creación de documentos se realiza utilizando un sistema de plantillas.
Las plantillas son ficheros del formato correspondiente al documento final a generar (HTML, \LaTeX{} o XML), que contienen un serie de etiquetas en su interior.
Durante la preparación del documento, estas etiquetas se sustituyen por la información concreta del ejercicio, generando una cuestión completa.

Gracias a las plantillas podemos almacenar información que se repite en cada problema, como por ejemplo el enunciado, sin necesidad de incluirla dentro del código.
Asimismo estas plantillas son modificables, gracias a lo cual podemos crear cuestionarios personalizados en cualquier estilo que permita el formato de destino.

La aplicación utiliza dos tipos de plantillas:
\begin{itemize}
	\item La plantilla base, que contiene la estructura general del documento.
	Se coloca una etiqueta en el punto en el que se insertarán las cuestiones.
	Contendrá, por ejemplo, las reglas de estilo CSS y las cabeceras en un documento HTML, o los paquetes a importar en un documento \LaTeX{}.
	\item Las plantillas de cada tipo individual de ejercicio.
	Contienen el enunciado y estructura de cada tipo y sub-tipo de cuestión, con etiquetas que indican donde insertar el número de problema y cada una de las partes del mismo.
	Podemos tener, por ejemplo, una etiqueta que indique donde va la expresión regular, o donde colocamos las imágenes.
\end{itemize}

\section{Estadísticas}
\subsection{Código}
Realizando un análisis del código fuente de la aplicación, obtenemos los siguientes valores estadísticos sobre las características del programa:
\begin{itemize}
	\item 17 paquetes
	\item 69 clases, de las cuales 22 son públicas y 8 son autogeneradas por JavaCC
	\item 521 métodos
	\item Más de 12.700 líneas de código
\end{itemize}

\subsection{Pruebas}
La aplicación utiliza JUnit 4 para realizar sus pruebas, y cuenta con un total de 149 pruebas, agrupadas en 18 clases.
Las pruebas excluyen las clases relacionadas con la interfaz gráfica, y aquellas autogeneradas (como por ejemplo las pertenecientes al paquete \ruta{es.ubu.inf.tfg.regex.parser}, generadas por JavaCC).

Las herramientas aplicadas indican una cobertura general superior al 65\%, con más del 90\% en las clases más relevantes.

\imagen{test_cobertura}{Resultados del análisis de cobertura de pruebas}