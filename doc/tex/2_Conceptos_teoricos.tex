\capitulo{2}{Conceptos teóricos}

\section{Generación de expresiones regulares}

% Descripción del algoritmo de generación de expresiones.

Realizaremos un análisis a partir de una muestra apreciable de expresiones regulares generadas al azar, para profundidades del árbol sintáctico de entre 1 y 6.

\subsection{Regiones aceptables}

Consideramos expresiones 'útiles' aquellas que contienen entre 2 y 6 símbolos distintos (sin incluir el vacío), y entre 3 y 15 estados distintos al generar a partir de la expresión un problema de tipo Aho-Sethi-Ullman o de construcción de subconjuntos.
Estos rangos corresponden con los problemas que consideramos apropiados para resolver de manera manual, y con lo permitido por la interfaz.
El generador es capaz de trabajar con rangos arbitrarios.
\\
El conocimiento de que profundidades de árbol corresponden con más expresiones generadas dentro de los rangos deseados nos permite establecer los límites óptimos para utilizar con los generadores de problemas.
\\
Este análisis pretende determinar cuantas de las expresiones regulares generadas aleatoriamente encajan dentro de los rangos deseados, y en que medida depende de la profundidad del árbol sintáctico.
La intuición nos dice que los árboles más profundos corresponderán con las expresiones más complejas (más símbolos y/o estados).

\subsubsection{Aho-Sethi-Ullman}

Para realizar las pruebas generaremos 100.000 expresiones regulares de cada tipo (conteniendo el símbolo vacío y sin contenerlo) para cada profundidad de entre 1 y 6.
Con cada una de las expresiones construiremos un problema de tipo Aho-Sethi-Ullman y comprobaremos cuantas de las expresiones generadas entran dentro de los rangos de símbolos y estados.

\imagen{ASU-NV}{Aho-Sethi-Ullman (no vacío)}
\imagen{ASU-V}{Aho-Sethi-Ullman (vacío)}

% Estudio de tiempos. Profundidad 7+ aumenta el tiempo de generación de los árboles.

Los resultados muestran que los árboles de profundidad 4 generan el mayor número de expresiones en la zona deseada.
También podemos ver claramente que las expresiones con árbol sintáctico de profundidad entre 4 y 6 generan una mayoría de expresiones regulares aceptables, tanto si incluimos el símbolo vacío como si no.

\subsubsection{Construcción de subconjuntos}

Para el análisis de problemas de construcción de subconjuntos 

\imagen{CS-NV}{Construcción de subconjuntos (no vacío)}
\imagen{CS-V}{Construcción de subconjuntos (vacío)}