\documentclass[a4paper,11pt,oneside]{memoir}

% Castellano
\usepackage[spanish]{babel}
\selectlanguage{spanish}
\usepackage[utf8]{inputenc}
\usepackage{placeins}

% Imagenes
\usepackage{graphicx}
\newcommand{\imagen}[2]{
	\begin{figure}[!h]
		\centering
		\includegraphics[width=0.9\textwidth]{#1}
		\caption{#2}
	\end{figure}
	\FloatBarrier
}

\graphicspath{ {./img/} }

% Capítulos
\chapterstyle{bianchi}
\newcommand{\capitulo}[2]{
	\setcounter{chapter}{#1}
	\setcounter{section}{0}
	\chapter*{#2}
	\addcontentsline{toc}{chapter}{#2}
	\markboth{#2}{#2}
}

% Apéndices
\renewcommand{\appendixname}{Apéndice}
\renewcommand*\cftappendixname{\appendixname}

\newcommand{\apendice}[1]{
	%\renewcommand{\thechapter}{A}
	\chapter{#1}
}

\renewcommand*\cftappendixname{\appendixname\ }

% Formato de portada
\makeatletter
\usepackage{xcolor}
\newcommand{\tutor}[1]{\def\@tutor{#1}}
\newcommand{\course}[1]{\def\@course{#1}}
\definecolor{cpardoBox}{HTML}{E6E6FF}
\def\maketitle{
  \null
  \thispagestyle{empty}
  % Cabecera ----------------
\noindent\includegraphics[width=\textwidth]{cabecera}\vspace{1cm}%
  \vfill
  % Título proyecto y escudo informática ----------------
  \colorbox{cpardoBox}{%
    \begin{minipage}{.8\textwidth}
      \vspace{.5cm}\Large
      \begin{center}
      \textbf{TFG del Grado en Ingeniería Informática}\vspace{.6cm}\\
      \textbf{\LARGE\@title{}}
      \end{center}
      \vspace{.2cm}
    \end{minipage}

  }%
  \hfill\begin{minipage}{.20\textwidth}
    \includegraphics[width=\textwidth]{escudoInfor}
  \end{minipage}
  \vfill
  % Datos de alumno, curso y tutores ------------------
  \begin{center}%
  {%
    \noindent\LARGE
    Presentado por \@author{}\\ 
    en Universidad de Burgos --- Curso \@course{}\\
    Tutor: \@tutor{}\\
  }%
  \end{center}%
  \null
  \cleardoublepage
  }
\makeatother


% Datos de portada
\title{PLQuiz}
\author{Roberto Izquierdo Amo}
\tutor{Dr. César Ignacio García Osorio}
\course{2013/2014}
\date{}

\begin{document}

\maketitle

\frontmatter

% Abstract en castellano
\renewcommand*\abstractname{Resumen}
\begin{abstract}
Desarrollo de una herramienta de escritorio que ayude a generar preguntas de test aleatorias (tipo quiz, cloze, de texto libre...) sobre problemas de algoritmos de análisis léxico.
El formato utilizado para generar las preguntas será importable a entornos virtuales de aprendizaje (Moodle), y permitirá su impresión en papel a través de la obtención de código \LaTeX .
\end{abstract}

\renewcommand*\abstractname{Descriptores}
\begin{abstract}
Generación de cuestionarios, análisis léxico, expresiones regulares, autómatas finitos.
\end{abstract}

\clearpage

% Abstract en inglés
\renewcommand*\abstractname{Abstract}
\begin{abstract}
Development of a desktop application that generates random test questions (quiz, cloze, free text...) about lexical analysis.
The format used to produce the questions should be compatible with virtual learning frameworks (such as Moodle), and allow for physical printing by generating \LaTeX\ code.
\end{abstract}

\renewcommand*\abstractname{Keywords}
\begin{abstract}
Questionnaire generation, lexical analysis, regular expresions, finite automata.
\end{abstract}

\clearpage

% Indices
\tableofcontents

\clearpage

\listoffigures

\clearpage

%\listoftables

%\clearpage

\mainmatter

\capitulo{1}{Objetivos del proyecto}
\capitulo{2}{Conceptos teóricos}

\section{Generación de expresiones regulares}

% Descripción del algoritmo de generación de expresiones.

Realizaremos un análisis a partir de una muestra apreciable de expresiones regulares generadas al azar, para profundidades del árbol sintáctico de entre 1 y 6.

\subsection{Regiones aceptables}

Consideramos expresiones 'útiles' aquellas que contienen entre 2 y 6 símbolos distintos (sin incluir el vacío), y entre 3 y 15 estados distintos al generar a partir de la expresión un problema de tipo Aho-Sethi-Ullman o de construcción de subconjuntos.
Estos rangos corresponden con los problemas que consideramos apropiados para resolver de manera manual, y con lo permitido por la interfaz.
El generador es capaz de trabajar con rangos arbitrarios.
\\
El conocimiento de que profundidades de árbol corresponden con más expresiones generadas dentro de los rangos deseados nos permite establecer los límites óptimos para utilizar con los generadores de problemas.
\\
Este análisis pretende determinar cuantas de las expresiones regulares generadas aleatoriamente encajan dentro de los rangos deseados, y en que medida depende de la profundidad del árbol sintáctico.
La intuición nos dice que los árboles más profundos corresponderán con las expresiones más complejas (más símbolos y/o estados).

\subsubsection{Aho-Sethi-Ullman}

Para realizar las pruebas generaremos 100.000 expresiones regulares de cada tipo (conteniendo el símbolo vacío y sin contenerlo) para cada profundidad de entre 1 y 6.
Con cada una de las expresiones construiremos un problema de tipo Aho-Sethi-Ullman y comprobaremos cuantas de las expresiones generadas entran dentro de los rangos de símbolos y estados.

\imagen{ASU-NV}{Aho-Sethi-Ullman (no vacío)}
\imagen{ASU-V}{Aho-Sethi-Ullman (vacío)}

% Estudio de tiempos. Profundidad 7+ aumenta el tiempo de generación de los árboles.

Los resultados muestran que los árboles de profundidad 4 generan el mayor número de expresiones en la zona deseada.
También podemos ver claramente que las expresiones con árbol sintáctico de profundidad entre 4 y 6 generan una mayoría de expresiones regulares aceptables, tanto si incluimos el símbolo vacío como si no.

\subsubsection{Construcción de subconjuntos}

Para el análisis de problemas de construcción de subconjuntos 

\imagen{CS-NV}{Construcción de subconjuntos (no vacío)}
\imagen{CS-V}{Construcción de subconjuntos (vacío)}
\capitulo{3}{Técnicas y herramientas}
\capitulo{4}{Aspectos relevantes del desarrollo del proyecto}

\section{Arquitectura de la aplicación}
\subsection{Estructura de paquetes}
La filosofía seguida a la hora de diseñar la estructura de la aplicación se centra en los conceptos de modularidad y reusabilidad.
Cada paquete pretende ser independiente, y las dependencias externas se reducen lo más posible.
Esto se traduce en un uso frecuente del patrón `fachada'.

Cada paquete principal tiene una estructura de dos niveles: el primer nivel contiene las clases `fachada' --- visibles al resto de la aplicación ---, mientras que el segundo (\emph{.datos}) contiene la lógica y estructuras de datos internas --- utilizadas solo por el paquete `padre' ---.

Podemos dividir los paquetes principales de la aplicación en tres clases:
\begin{itemize}
	\item Paquete de interfaz gráfica (\emph{es.ubu.inf.tfg.ui}).
	Contiene las clases que definen la apariencia y comportamiento de la interfaz de usuario.
	Es, por pura necesidad, el paquete con más dependencias externas, dependiendo tanto del paquete que define la representación de documentos como de los paquetes que definen cada clase de problemas.
	\item Paquete de construcción de documentos (\emph{es.ubu.inf.tfg.doc}).
	Dependen, por necesidad, de los paquetes que definen las clases de problemas que van a representar.
	\item Paquetes de problemas (\emph{es.ubu.inf.tfg.regex.thompson} y \emph{es.ubu.inf.tfg.regex.asu}).
	Contienen las clases encargadas de representar las estructuras internas del problema, sus soluciones, y aquellas necesarias para generarlos.
	Dependen únicamente del paquete de representación de expresiones regulares, y son independientes el uno del otro.
	\item Paquete de representación y procesamiento de expresiones regulares (\emph{es.ubu.inf.tfg.regex.parser} y \emph{es.ubu.inf.tfg.regex.datos}).
	Contienen las clases necesarias para procesar expresiones regulares a partir de cadenas de caracteres, representarlas y operar sobre ellas.
	Son la base sobre la cual trabajan los problemas.
\end{itemize}

\imagen{diagrama_paquetes}{Diagrama de paquetes y sus dependencias}

Como podemos ver, los paquetes encargados del procesamiento de expresiones regulares son la base del problema.
Junto con los paquetes de problemas forman el núcleo de la aplicación, y pueden ser utilizados de manera independiente, al carecer de dependencias externas.

Sobre estos paquetes centrales se aposentan los paquetes de representación de documentos, que toman los datos de los documentos y les dan un formato adecuado.
En un nivel superior esta la interfaz de usuario, que toma el conjunto de datos obtenido del resto de paquetes y lo presenta al usuario.

Esta estructura en niveles significa que podemos tomar cualquier paquete, y utilizarlo de manera completamente independiente, siempre que dispongamos de aquellos que se encuentran en niveles inferiores.

\subsection{Patrones de diseño}
\capitulo{5}{Trabajos relacionados}
\capitulo{6}{Líneas de trabajo futuras}

\section{Algoritmos de generación de árboles de expresión regular}
El sistema de generación de expresiones regulares actual implementa el método \emph{full} descrito por John R. Koza \cite{koza92}.
Dicho método genera árboles en los que todas las hojas tienen la misma altura.

Este método permite generar expresiones de forma rápida y resulta relativamente simple de implementar, dado que los árboles no varían en profundidad.

Como desventaja, el método \emph{full} genera un rango de expresiones limitado.
Esta característica supone un problema a la hora de utilizar las expresiones para generar problemas.

Si agrupamos los problemas según los valores de las características que encontramos para generarlos, podemos ver que la distribución de los mismos no resulta uniforme, sino que se generan muchos problemas con ciertos valores y muy pocos o ninguno con otros.
Esto resulta problemático si presentamos una interfaz de usuario que permita introducir valores arbitrarios.

En la literatura encontramos definidos otros dos posibles algoritmos, \emph{grow} y \emph{half-and-half} \cite{koza92}.
La implementación de uno de estos algoritmos (o de algún otro equivalente) supondría una mejora del sistema de generación y permitiría comprobar como de equilibrada está la distribución de resultados.

Las mejoras en el algoritmo de generación de árboles afectarían también a los algoritmos de búsqueda, permitiendo tal vez utilizar métodos más rápidos o eficientes, y generar problemas más complejos.

\section{Mejoras a la interfaz gráfica}
\subsection{Opción de guardado y carga de cuestionarios}
En la versión actual de la aplicación, los cuestionarios pueden exportarse pero no importarse.
Esto quiere decir que una vez se cierra el programa no puede continuar trabajandose sobre las preguntas que teníamos, a menos que se cree un cuestionario nuevo y se añadan las preguntas a mano.

Se proponen dos posibles implementaciones de un sistema de guardado y carga:
\begin{itemize}
	\item El método más evidente es el guardado de los cuestionarios como ficheros separados, ya sea en texto plano, \emph{XML} o cualquier otro formato.
	Lo único que necesitamos conocer para reconstruir un problema es la expresión regular que lo define, y el tipo y sub-tipo de problema con el que se resuelve.
	Por lo tanto la cantidad de información a almacenar es muy reducida, y el sistema que la almacene y recupere sería simple de implementar.
	\item Un método más complejo, pero que no requiere guardar ficheros separados, consistiría en la lectura de un fichero exportado y su comparación con la plantilla de la que se generó.
	De esta manera podemos localizar la expresión y extraerla.
	Una vez obtenidas las expresiones regulares e identificado el tipo de problema al que pertenece cada una, la reconstrucción del cuestionario es trivial.
\end{itemize}

\subsection{Opción de deshacer cambios}
La aplicación no permite `volver atrás' si eliminamos parte del trabajo realizado, como por ejemplo eliminando un problema del cuestionario, o pulsando el botón de `Generar' en un problema que queríamos conservar.
Una opción de `deshacer' evitaría problemas al usuario y aumentaría la usabilidad del programa.

\subsection{Opción de reordenar cuestiones}
Las cuestiones se muestran en el orden en que se añadieron, sin permitir cambios.
Si se pretende añadir una cuestión entre varias existentes primero deben eliminarse las que se encuentran por debajo de la nueva posición.

La implementación del cambio de orden es sencilla, ya que el documento que la aplicación utiliza internamente ya almacena los problemas de manera ordenada.
Es necesario añadir un sistema que permita especificar un nuevo orden en el documento o una nueva posición para un problema dado.
Una posibilidad sería asociar cada problema con un número que indique su posición.

Se proponen varias implementaciones de cara a la interfaz gráfica:
\begin{itemize}
	\item La manera más sencilla de implementar la ordenación sería añadir dos botones al `frame' que representa el problema, uno para mover el problema hacia arriba y otro para moverlo hacia abajo.
	Es una implementación con usabilidad limitada, ya que tenemos que pulsar el botón tantas veces como posiciones queramos mover el problema.
	\item Una implementación sencilla y más usable seria el añadir un cuadro de texto o lista desplegable al `frame' del problema, que nos permita seleccionar una nueva posición para el problema.
	\item La implementación más compleja, pero más adecuada desde el punto de vista de la usabilidad, es la implementación de un sistema de `drag and drop'.
	Es decir, que la aplicación nos permita arrastrar un problema hasta su nueva posición.
	El problema de esta idea es que \emph{Swing} no soporta estas operaciones, y añadirlas supondría un esfuerzo de desarrollo considerable.
\end{itemize}

\section{Mejora del sistema de plantillas}
\subsection{Personalización de plantillas}
El sistema de plantillas de la aplicación permite la modificación de las mismas para conseguir documentos personalizados.
Sin embargo, la personalización de las plantillas requiere modificación y empaquetado del proyecto, o modificación directa de los ficheros del \emph{.jar}.

Una posible mejora sería la opción de proveer plantillas personalizadas.
Estas plantillas se seleccionarían al realizar la exportación del cuestionario, o desde un menú de opciones en la propia aplicación.
Las plantillas por defecto seguirían estando disponibles, en caso de que no se haya creado una propia y como ejemplo.

Con esta mejora no solo permitimos personalizar el estilo, sino también el contenido.
El usuario puede añadir o eliminar etiquetas, haciendo que el problema disponga de más o menos información.

\subsection{Integración de un motor de plantillas}
La herramienta de plantillas que la aplicación utiliza es propia, simple, y se basa en sustitución mediante expresiones regulares.
Una posible mejora es la introducción de un motor de plantillas externo en forma de librería, que aumente las opciones a la hora de trabajar con las mismas, o que las simplifique.

\section{Más tipos de cuestiones}
Una de las mayores limitaciones de la aplicación es el reducido número de tipos de problemas de los que dispone, y a partir de los cuales se generan los cuestionarios.
Una de las mejoras más obvias es una ampliación en el tipo de cuestiones disponibles.

\subsection{Integración con BURGRAM}
La aplicación BURGRAM\footnote{Carlos Gómez Palacios, Enero 2008, Universidad de Burgos}, desarrollada como proyecto de fín de carrera, permite la resolución de problemas asociados a gramáticas tipos \emph{LL}, \emph{LR}, \emph{SLR} y \emph{LALR}.
Aunque estos tipos de problemas son bastante diferentes a los que la aplicación resuelve actualmente, se enmarcan también dentro del área de procesamiento del lenguaje.
Los consideramos, por lo tanto, un añadido interesante a la aplicación.

La aplicación BURGRAM trata la resolución de los problemas paso a paso y a partir de una gramática dada.
El procesamiento y resolución del problemas están ya resueltos, por lo que su integración consistiría en adaptar la interfaz, añadir los distintos tipos al modelo de documentos y preparar las plantillas.
Idealmente la integración trataría a BURGRAM como una librería externa, sin incluir directamente su código dentro del proyecto.

\subsection{Compatibilidad con ficheros\emph{GIFT}}
El formato \emph{GIFT} es el usado por la plataforma de aprendizaje virtual Moodle para el almacenamiento de cuestiones en formato de texto plano.
Los tipos de preguntas disponibles son:
\begin{itemize}
	\item Selección múltiple.
	\item Verdadero o falso.
	\item Respuesta corta.
	\item Rellenar huecos.
	\item Cuestiones numéricas.
\end{itemize}

La compatibilidad con este formato nos permitiría importar o incluir con la aplicación un conjunto de preguntas de tipo teórico, que podrían insertarse en los cuestionarios.
Esto permitiría la creación de cuestionarios y examenes más completos, con preguntas de test, teóricas y resolución de problemas.

\appendix

\apendice{Generación de expresiones regulares}

Trabajar con una expresión regular dentro de la aplicación implica traducirla a una estructura de datos adecuada.
Contando con que la expresión es introducida como una cadena de caracteres, utilizaremos un analizador léxico y sintáctico para convertirla en un árbol binario.
Los algoritmos de generación de árboles son un tema muy estudiado dentro de la programación genética, lo cuál nos permite generar la expresión directamente en forma de árbol implementando un algoritmo ya conocido.

El algoritmo concreto implementado es el del método \emph{full} \cite{koza92}.
Este método involucra la creación de árboles cuya profundidad viene definida por la longitud del camino entre un extremo cualquiera y la raíz.
Esto quiere decir que generaremos árboles binarios en los que cada cada nodo no hoja tiene 1 o 2 hijos, y en el que todas las hojas se encuentran a la misma profundidad.

El algoritmo funciona restringiendo la selección de etiquetas para los nodos generados en función de su profundidad.
Un nodo símbolo, por ejemplo, solo estará permitido en la profundidad máxima, mientras que un nodo cerradura estará permitido en cualquier profundidad excepto la máxima.
Tomaremos consideraciones adicionales a la hora de generar los hijos de un nodo, ya que el número de los mismos variará en función de la operación que contenga el padre.
Un nodo cerradura, por ejemplo, tiene un solo hijo, mientras que un nodo concatenación tendrá dos.

\section{Generación de problemas}

A la hora de generar problemas tomamos tres criterios:
\begin{itemize}
	\item El número de símbolos presentes en la expresión regular (sin contar el que representa la palabra vacía).
	\item El número de estados en su función de transición.
	\item La presencia o ausencia del símbolo que representa la palabra vacía.
\end{itemize}
Tanto el número de símbolos como la presencia del símbolo vacío dependen directamente de la expresión regular a partir de la cual construimos el problema.
El número de estados de la función de transición será el mismo para cualquier problema de un mismo tipo que comparta la misma expresión regular.
Podemos decir, por lo tanto, que las características que buscamos en un problema dependen exclusivamente de la expresión regular con la cuál lo construimos.

La generación de problemas por lo tanto consistirá en generar expresiones regulares cuyo problema asociado cumpla las características pedidas.
Siendo el objetivo la generación de problemas con unas características dadas, deben determinarse los algoritmos de búsqueda a utilizar y los parámetros óptimos para los mismos.

\section{Coste de generación}

Dado que los árboles para las expresiones regulares son árboles binarios, sabemos que el número máximo de nodos está acotado superiormente por la función:

\begin{equation*}
\text{\em número de nodos} \leq 2^\text{\em profundidad} - 1
\end{equation*}

La generación de una expresión regular crea los nodos del árbol de la expresión uno a uno, por lo tanto el coste de generar una expresión crecerá con el número de nodos en su árbol sintáctico.

Para comprobar estos datos de manera experimental obtendremos los tiempos medios de generación de 100.000 expresiones regulares para cada profundidad entre 0 y 8, y comparamos su crecimiento con el número máximo de nodos asociado a cada profundidad.

\imagen{profundidad}{Tiempo de generación de expresiones y número de nodos}

Podemos ver como el tiempo de generación de una expresión regular depende del número de nodos que contenga el árbol sintáctico de la misma y, por lo tanto, de la profundidad del mismo.
Por lo tanto, será preferible generar expresiones lo menos profundas posible, en concreto, limitándonos a una profundidad máxima de 6.

Destacamos que los valores del tiempo dependen de la máquina en que se realizan las pruebas.
Este análisis pretende analizar el crecimiento de la función tiempo, no sus valores exactos.

\section{Regiones aceptables}

Consideramos expresiones `útiles' aquellas que contienen entre 2 y 6 símbolos distintos (sin incluir el vacío), y entre 3 y 15 estados distintos al generar a partir de la expresión un problema de tipo Aho-Sethi-Ullman o de construcción de subconjuntos.
Estos rangos corresponden con los problemas que consideramos apropiados para resolver de manera manual, y con lo permitido por la interfaz.
El generador es capaz de trabajar con rangos arbitrarios, pero los problemas que aparecen con expresiones fuera de estos rangos son o demasiado complejos para resolverlos de manera manual, o completamente triviales.

El conocimiento de qué profundidades de árbol corresponden con más expresiones generadas dentro de los rangos deseados nos permite establecer los límites óptimos para utilizar con los generadores de problemas.

Este análisis pretende determinar cuantas de las expresiones regulares generadas aleatoriamente encajan dentro de los rangos deseados, y en qué medida depende de la profundidad del árbol sintáctico.
La intuición nos dice que los árboles más profundos corresponderán con las expresiones más complejas (más símbolos y/o estados en la tabla de transición).

\subsection{Aho-Sethi-Ullman}

Para realizar las pruebas generaremos 100.000 expresiones regulares conteniendo $ \epsilon $, y otras tantas sin contenerlo, para cada profundidad de árbol de entre 1 y 6.
Con cada una de las expresiones construiremos un problema de tipo Aho-Sethi-Ullman y comprobaremos cuantas de las expresiones generadas entran dentro de los rangos de símbolos y estados que consideramos aceptables.

\imagen{prof-ASU-NV}{Aho-Sethi-Ullman (sin $ \epsilon $) aceptables según profundidad}
\imagen{prof-ASU-V}{Aho-Sethi-Ullman (con $ \epsilon $) aceptables según profundidad}

Los resultados muestran que los árboles de profundidad 4 generan el mayor número de expresiones en la zona deseada.
También podemos ver claramente que las expresiones con árbol sintáctico de profundidad entre 3 y 6 generan una mayoría de expresiones regulares aceptables, tanto si incluimos el símbolo vacío como si no.

Por lo tanto el método de búsqueda para problemas de Aho-Sethi-Ullman utilizará expresiones de profundidad entre 3 y 6.

\subsection{Construcción de subconjuntos}

Para el análisis de problemas de construcción de subconjuntos repetimos el mismo proceso que en el apartado anterior, generando expresiones regulares para cada combinación de tipo y profundidad, y construyendo problemas de construcción de subconjuntos con ellas.

\imagen{prof-CS-NV}{Construcción de subconjuntos (sin $ \epsilon $) aceptables según profundidad}
\imagen{prof-CS-V}{Construcción de subconjuntos (con $ \epsilon $) aceptables según profundidad}

Según los resultados podemos ver que las expresiones dentro del rango tienen mayoritariamente profundidades de entre 1 y 5, y que las expresiones de profundidad 6 están completamente fuera del rango.

Por lo tanto, el método de búsqueda para problemas de construcción de subconjuntos utilizará expresiones de profundidad entre 1 y 5.

\section{Distribución de resultados aceptables}

Partiendo de los resultados que encontramos dentro de la región aceptable, es importante identificar como se distribuyen esos resultados.
Asumimos que no todas las combinaciones de símbolo y estado van a aparecer con la misma frecuencia dentro de la región aceptable.

Dado que la frecuencia de aparición de expresiones en la región aceptable varía según la profundidad, examinaremos cada profundidad utilizada por separado.
De esta manera podremos determinar si ciertas combinaciones de símbolo y estado son más probables en ciertas profundidades.

Los datos experimentales utilizados son los mismos que en los apartados anteriores.

\subsection{Aho-Sethi-Ullman}

Agrupando los datos según profundidades podemos ver nuevamente las agrupaciones según profundidad.
Más interesante resulta que las expresiones generadas se agrupan, independientemente de la profundidad, alrededor de los mismos puntos.

Es interesante notar que, aunque las agrupaciones no dependen de la profundidad, las profundidades mayores parecen agruparse más hacia la izquierda.
Esto implicaría que las profundidades mayores generan problemas con menos estados y símbolos o, más probablemente, que nuestro rango aceptable es demasiado restringido para ver la perspectiva completa.

\imagen{dist-ASU-NV}{Distribución de Aho-Sethi-Ullman (sin $ \epsilon $)}
\imagen{dist-ASU-V}{Distribución de Aho-Sethi-Ullman (con $ \epsilon $)}

Vemos que el comportamiento es similar entre expresiones que contienen el elemento vacío y aquellas que no.
Esto indica, de manera bastante clara, que ciertas combinaciones de número de símbolos y estados son mucho más probables.
O, por otra parte, que en las profundidades con las que estamos trabajando tienden a generarse expresiones con unas características específicas.

Por ejemplo, es posible que la combinación de 3 símbolos y 8 a 14 estados se dé con alta probabilidad en profundidades de 6 o mayores, o que necesite expresiones regulares con árboles que tengan hojas a distintas profundidades.

\subsection{Construcción de subconjuntos}

En los problemas de construcción vemos los mismos problemas que en los de Aho-Sethi-Ullman.
Las expresiones regulares se agrupan en torno a unas ciertas combinaciones de número de símbolos y estados, aunque en este caso las combinaciones son diferentes que en el apartado anterior.

\imagen{dist-CS-NV}{Distribución de construcción de subconjuntos (sin $ \epsilon $)}
\imagen{dist-CS-V}{Distribución de construcción de subconjuntos (con $ \epsilon $)}

Es interesante remarcar que las agrupaciones parecen ampliarse hacia la derecha.
Es decir, cuanto más elevado el número de símbolos, más variedad de estados generan los problemas.

\section{Algoritmos de búsqueda}

Una vez definidos los parámetros con los cuales vamos a generar las expresiones, queda pendiente encontrar un algoritmo de búsqueda que nos permita encontrar los problemas dados.

Es probable que dos problemas generados a partir de expresiones regulares tengan características similares, pero no tenemos ninguna garantía al respecto.
Es decir, es posible que dos expresiones vecinas tengan características bastante diferentes.
Descartamos, por lo tanto, algoritmos de búsqueda como el recocido simulado, que se desplazan por el espacio de búsqueda de vecino a vecino.

\subsection{Búsqueda aleatoria}

El algoritmo de búsqueda aleatoria consiste en generar soluciones completamente al azar hasta encontrar la correcta.
Lo tomamos como posibilidad dado que resulta muy sencillo de implementar, y de que resultará eficiente en el caso de búsquedas sencillas.

% ASU, CS

\subsection{Algoritmo genético}

% ASU, CS, Reproducción o mutación

\section{Conclusiones}

% puede ser que la forma de los árboles de expresiones (nodos hoja con misma profundidad) favorezcan ciertas combinaciones de simbolos / estados y desfavorezcan otras.
\apendice{Manuales}

\section{Manual de usuario}

\subsection{Requerimientos}
La aplicación cuenta con los siguientes requerimientos:
\begin{itemize}
	\item Java 8\footnote{http://www.oracle.com/technetwork/java/javase/downloads/jre8-downloads-2133155.html}
	\item Windows
	\begin{itemize}
		\item Windows 8 (Desktop) / Windows 7 / Windows Vista SP2 / Windows Server 2008 / Windows Server 2012 (64-bit)
		\item RAM: 128 MB; 64 MB en Windows XP (32-bit)
		\item Espacio en disco: 124 MB
	\end{itemize}
	\item OS X
	\begin{itemize}
		\item Procesador Intel
		\item Mac OS X 10.7.3 (Lion) o posterior
		\item Privilegios de administrador para la instalación
	\end{itemize}
	\item Linux
	\begin{itemize}
		\item Oracle Linux 5.5+ / Red Hat Enterprise Linux 5.5+ / Ubuntu Linux 10.04+ / Suse Linux Enterprise Server 10 SP2+
	\end{itemize}
\end{itemize}

Los requerimientos del sistema vienen dados por la instalación de Java\footnote{http://java.com/en/download/help/sysreq.xml}.

\subsection{Instalación y ejecución}
\subsubsection{Instalación}
La aplicación se compone de un único archivo \emph{jar} ejecutable.
Por lo tanto, el método de instalación se reduce a copiar y pegar el fichero a la localización deseada.

\subsubsection{Ejecución}
Existen dos métodos posibles de ejecución.
\begin{itemize}
	\item Directamente haciendo \emph{doble-click} sobre el fichero \emph{jar}.
	Asumiendo que el \emph{path} de Java se encuentre correctamente configurado, el programa comenzará a ejecutarse inmediatamente.
	\item Mediante línea de comandos, navegando hasta la carpeta que contiene el fichero \emph{jar} y ejecutando el comando
	\begin{verbatim}
	java -jar PLQuiz-[versión].jar
	\end{verbatim}
\end{itemize}

\subsection{Resolución de problemas}
La aplicación dispone de dos tipos distintos de problema con los que componer cuestionarios.
Cada uno de estos tipos dispone de tres sub-tipos adicionales, que se centran en aspectos distintos de un mismo problema general.
Los tipos y sub-tipos disponibles son los siguientes:
\begin{itemize}
	\item Problemas de aplicación del algoritmo de Aho-Sethi-Ullman
	\begin{itemize}
		\item Problemas de construcción de árbol.
		El objetivo es construir el árbol sintáctico correspondiente a la expresión regular dada, o elegir el correcto de entre los proporcionados.
		\item Problemas de etiquetado de árbol.
		Tienen como objetivo el etiquetado de cada nodo del árbol sintáctico correspondiente a una expresión regular con sus correspondientes conjuntos \emph{primerapos} y \emph{últimapos}, indicando además si son o no anulables.
		\item Problemas de construcción de tablas.
		El sub-tipo de problema más completo dentro de esta clase, asume la ejecución de los otros dos sub-tipos como paso previo a su resolución.
		Tiene como objetivo rellenar las tablas \emph{siguientepos} y de transición.
	\end{itemize}
	\item Problemas de aplicación del algoritmo de McNaughton-Yamada-Thompson
	\begin{itemize}
		\item Problema de construcción de autómata.
		Tiene como objetivo construir el autómata finito correspondiente con la expresión regular dada, o elegir el correcto de entre los dados.
		\item Problema de resolver expresión.
		El objetivo de este sub-tipo de problemas es la obtención de una tabla de transición a partir de una expresión regular dada.
		Asume la construcción del autómata como paso intermedio, combinando los otros dos sub-tipos de problema.
		\item Problema de resolver autómata.
		Consiste en obtener la tabla de transición de un autómata finito dado.
	\end{itemize}
\end{itemize}

\subsubsection{Añadir un problema}
El añadido de problemas se realiza mediante los controles situados en la parte inferior izquierda de la ventana.
La lista desplegable permite seleccionar el tipo de problema deseado.

Una vez escogido el tipo, el botón marcado con `+' añadirá un problema de ese tipo al final de la lista de problemas actuales, situada en la parte izquierda de la ventana.

\imagen{manual_anadir}{Controles de añadido de problemas}

Existe la posibilidad de añadir bloques de múltiples problemas de cualquiera de los tipos al mismo tiempo, generando automáticamente sus contenidos.
El método para realizarlo se detalla en la sección \emph{generación de problemas}.

\subsubsection{Formato de la expresión regular}
La expresión regular debe concordar con un formato especifico para ser correctamente reconocida por el parser.
\begin{itemize}
	\item Todo paréntesis de apertura `(' debe ir acompañado por un paréntesis de cierre `)'.
	\item Los símbolos utilizados deben ser una letra minúscula (a-z) o el símbolo `\$'.
	\item El caracter que representa la cadena vacía ($ \epsilon $) se representa con la letra `E' o con el caracter unicode \emph{\\u03B5}.
	\item La operación de cierre se representa con el símbolo `*', y va siempre precedida de un símbolo o de $ \epsilon $.
	\item La operación de concatenación se representa con el símbolo `.' o con el caracter unicode \emph{\\u2027}.
	\item La operación de unión se representa con el símbolo `$ | $'.
	\item Se admiten espacios en blanco en el interior de la expresión.
\end{itemize}

\subsubsection{Eliminar un problema}
La eliminación de un problema concreto se realiza mediante el botón situado a la izquierda de la caja de texto que contiene la expresión regular, identificado con el símbolo `-'.

\imagen{manual_eliminar}{Contról de eliminación de problemas}

La otra opción de eliminación de problemas permite eliminar el cuestionario completo y empezar de cero.
Resulta accesible desde el menú `Archivo', tomando la opción de `Documento en blanco'.

\imagen{manual_nuevo}{Control de generación de nuevo documento en blanco}

Es importante remarcar que la eliminación de los problemas es \emph{definitiva}.

\subsubsection{Tipos y sub-tipos de problema}

\subsection{Generación de problemas}

\subsubsection{Características del problema}
A la hora de generar un problema tenemos la opción de definir ciertas características del mismo, incluyendo:
\begin{itemize}
	\item Si la expresión regular asociada incorporará $ \epsilon $.
	\item El número de símbolos contenidos en la expresión regular, empezando por la `a' y sin incluir $ \epsilon $.
	Por ejemplo, la expresión $ ((a|b)*c)|\epsilon $ contendría tres símbolos.
	\item El número de estados en la tabla de transición del problema resuelto.
\end{itemize}
Las características se definen en la parte inferior del recuadro del problema.

\imagen{manual_caracteristicas}{Características seleccionadas y problema generado}

\subsubsection{Generación de expresiones}
Una vez definidas las características del problema deseado, pulsar el botón `Generar' comenzará la búsqueda de una expresión regular que produzca dicho problema.
Mientras la aplicación realiza la búsqueda, se mostrará una barra de progreso en la parte inferior del recuadro del problema, y el botón de `Generar' se transformará en el botón de `Cancelar'.
Dicho botón puede usarse durante la búsqueda para cancelar la operación.

\imagen{manual_generacion}{Generación de problema en curso}

Una vez generado el problema, la solución se muestra automáticamente de acuerdo con el sub-tipo de problema seleccionado.

La generación de problemas se realiza en un proceso aislado, por lo que es posible interactuar con la interfaz durante el proceso, o generar múltiples problemas al mismo tiempo.

\subsubsection{Generación de bloques de problemas}
La aplicación no está limitada a la generación de problemas individuales, sino que puede generar un número arbitrario de problemas de cualquier tipo.
Para evitar la generación de un gran número de problemas similares, la aplicación permite introducir un rango de variación, dentro de los cuales se encontrarán las características de los problemas generados.

El acceso a la herramienta de generación de problemas en bloque se encuentra en el menú `Archivo', seleccionando la opción `Generar bloque de problemas'.

\imagen{manual_menu_bloque}{Opción de menu para generación de problemas en bloque}

Dentro de la interfaz de generación de problemas podemos especificar el número de problemas de cada tipo principal que queremos generar.
Cada tipo de problemas permitirá seleccionar un sub-tipo, al cuál pertenecerán todos.

La selección de características deseadas tiene los mismos efectos que en la interfaz princial, con la diferencia de que permite especificar una variación.
Los problemas generados en este bloque se ajustarán al rango definido entre el valor dado a la característica menos la variación, y el valor más la variación.

\imagen{manual_bloque}{Interfaz de generación de problemas en bloque}

Al igual que la generación de problemas normal, la generación en bloque muestra una barra de progreso y permite la cancelación.

Una limitación de este método es que todos los problemas de un tipo dado que se generen pertenecerán al mismo sub-tipo.

\subsection{Exportado de cuestionarios}
El exportado del cuestionario actual se realiza mediante el menú `Exportar' de la barra de herramientas, seleccionando el formato deseado.
El cuestionario exportado no mostrará siempre el mismo aspecto que en la vista previa, y puede variar según el formato, de manera que la presentación sea la más adecuada en cada caso.

\imagen{manual_exportar}{Menu exportar, con todas las opciones de exportado de cuestionarios}

\subsubsection{Vista previa}
A diferencia del resto de formatos, la traducción de cuestionarios al formato de vista previa no está disponible al usuario, y se utiliza únicamente para la vista que aparece en el lado derecho de la interfaz gráfica.
El formato de salida es \emph{HTML} con estilos \emph{CSS} (aunque limitados), aprovechando la capacidad de los componentes \emph{Swing} para mostrarlo.

Dado que el contenido de un mismo cuestionario puede variar para los distintos formatos de salida, la vista previa mostrará siempre aquel que muestre más información.
Por ejemplo, si un cuestionario puede exportarse mostrando varias alternativas posibles además de la solución real, y con únicamente la solución, la vista previa mostrará las alternativas.

\subsubsection{Moodle \emph{XML}}
Esta traducción genera un cuestionario \emph{cloze} en formato \emph{XML} compatible con la plataforma de aprendizaje virtual Moodle.
El objetivo de un cuestionario \emph{cloze} es la elección de la solución correcta de entre varias posibles para cada pregunta.

Las imágenes de los cuestionarios Moodle no se almacenan de manera separada, sino que se insertan directamente en el documento \emph{XML}.
Para ello codificamos la imagen en forma de cadena de caracteres de 64 bits, y la incrustamos con etiquetas de archivo dentro del documento.

El importado del cuestionario \emph{cloze} se realiza en el menú `Banco de preguntas' de Moodle, sección `importar', y eligiendo el formato `Formato Moodle XML'.

\imagen{manual_importar_moodle}{Menu de importación de cuestionarios Moodle}

Una vez completado el proceso de importado, podemos ver las cuestiones individuales que se han añadido al banco de preguntas.
La aplicación marca cada cuestión generada con una identificación, indicando el tipo y sub-tipo de cada una.
Desde este menú podemos revisar las cuestiones, accediendo a una vista previa de las mismas, y eliminarlas del bando de preguntas.

\imagen{manual_moodle_banco}{Vista de cuestiones añadidas al banco de preguntas de Moodle}

\subsubsection{\LaTeX{}}
Los documentos formato \LaTeX{} pretenden ser examenes o cuestionarios imprimibles, por lo cual su formato tiende a ser diferente al de los cuestionarios virtuales.
La mayor diferencia es que los cuestionarios \emph{XML} se centran siempre en ofrecer varias opciones de respuesta, mientras que un cuestionario escrito puede exigir directamente la resolución del problema.

Por defecto el formato \LaTeX{} genera imágenes en formato \emph{JPG}, guardando el documento y las imágenes de manera separada en el directorio indicado.
El documento incrustará las imágenes directamente al ser compilado.
Los nombres dados a las imágenes son automáticos y únicos para cada imagen.

\subsubsection{\LaTeX{} con imágenes \emph{Graphviz}}
Este modo de exportado es idéntico al modo \LaTeX{} por defecto, salvo en el modo en que exporta las imágenes.
En lugar de generar la imagen directamente, lo que se exportan son programas en formato \emph{dot} para cada imagen, que pueden ejecutarse utilizando la herramienta \emph{Graphviz}.

Una vez generadas las imágenes, y sin importar el formato en que \emph{Graphviz} las emita, la compilación del documento \LaTeX{} es capaz de incluirlas directamente.

\section{Manual del programador}

\subsection{Requerimientos}
La preparación del proyecto para su desarrollo y empaquetado requiere obligatoriamente las siguientes herramientas:
\begin{itemize}
	\item Entorno de desarrollo Java 8 (Java JDK 8)\footnote{http://www.oracle.com/technetwork/java/javase/downloads/jdk8-downloads-2133151.html}
	\item Apache Maven\footnote{http://maven.apache.org/} versión 3 o posterior
\end{itemize}

Opcionalmente puede trabajarse con Eclipse como \emph{IDE}, con el \emph{plugin} de \emph{JavaCC}\footnote{http://eclipse-javacc.sourceforge.net/}.
El \emph{plugin} de Maven es opcional.
Java 8 exige Eclipse versión Luna o posterior.

\subsection{Preparación del proyecto}
\subsubsection{Descarga desde control de versiones}
El proyecto puede obtenerse directamente o a traves de su repositorio de Github\footnote{https://github.com/RobertoIA/PLQuiz}. Puede descargarse como un fichero comprimido mediante la opción `Download ZIP', utilizando uno de los clientes propietarios de Github con la opción `Clone in Desktop', o mediante la linea de comandos de git utilizando la URL de clonado\footnote{https://github.com/RobertoIA/PLQuiz.git}.

\imagen{manual_github}{Opciones de descarga desde control de versiones}

\subsubsection{Resolución de dependencias}
La resolución de dependencias se realiza directamente mediante Maven.
Si no va a utilizarse Eclipse como IDE, puede usarse el siguiente comando en el directorio que contenga el fichero \emph{pom.xml}:
\begin{verbatim}
	mvn dependency:resolve
\end{verbatim}
En caso de utilizarse Eclipse, este comando resulta redundante, y el procedimiento a seguir se detalla en la versión siguiente.

\subsubsection{Trabajando con Eclipse}
En caso de utilizarse Eclipse como entorno de desarrollo, Maven nos permite realizar la resolución de dependencias y la conversión del proyecto a proyecto de Eclipse en un solo paso.
Esto se realiza con el plugin de Eclipse para Maven (y no viceversa), que Maven se encargará de descargar automáticamente si no hemos utilizado antes.

El siguiente comando debe ejecutarse en el directorio que contenga el fichero \emph{pom.xml}:
\begin{verbatim}
	mvn eclipse:eclipse
\end{verbatim}
Una vez ha terminado, podemos importar el proyecto desde eclipse con el menú `File', opción `Import'.
La opción a utilizar es `Existing Projects into Workspace', a la que solo tenemos que indicarle donde se encuentra almacenado el proyecto.

\subsubsection{Compilación y empaquetado}
La preparación de la aplicación para su uso se realiza con Maven, que se encarga de controlar las versiones, empaquetar las librerias necesarias y realizar los test.
Para realizar el empaquetado de manera normal, ejecutaremos el siguiente comando en el directorio raíz del proyecto, donde encontremos el fichero \emph{pom.xml}.
\begin{verbatim}
	mvn package
\end{verbatim}
En caso de querer saltarnos la realización de test podemos utilizar el siguiente comando, en el mismo directorio que el anterior.
\begin{verbatim}
	mvn package -Dmaven.test.skip=true
\end{verbatim}
El fichero \emph{.jar} generado se encontrará en la carpeta `target', situada en la raíz del proyecto.
\apendice{Planificación por \emph{sprint}}

Esta sección detalla el trabajo realizado en cada \emph{sprint} del proyecto. Cada \emph{sprint} dura dos semanas, y corresponde con una serie de objetivos planteados y unos resultados conseguidos.
Al finalizar se realiza la entrega de una versión funcional de la aplicación.

Se incluye adicionalmente un \emph{sprint} inicial que transcurre entre la asignación del proyecto y el comienzo oficial.

\subsection{Prototipo --- hasta el 28 de febrero}

\subsubsection{Objetivos}
Desarrollo de un prototipo básico de la aplicación, como base a partir de la cual realizar las decisiones sobre el diseño y la funcionalidad de la aplicación final.

El prototipo debe incluir una interfaz gráfica simple, y debe permitir añadir y resolver problemas básicos. 
El código se desarrolla con la única intención de demostrar funcionalidad, y será desechado al final de la iteración.

\subsubsection{Desarrollo}

\subsubsection{Resultados}
Al final de esta iteración contamos con una aplicación de interfaz gráfica que incluye un primer esbozo de los elementos que incluirá la aplicación final.
Algunos de estos elementos no son funcionales, pero se incluyen para referenciar funcionalidad futura.

El prototipo es capaz de procesar expresiones regulares a partir de cadenas de caracteres, extraer un árbol sintáctico de la misma y utilizarlo para obtener las tablas de siguiente posición y transiciones y demás datos correspondientes con el algoritmo de Aho-Sethi-Ullman

\subsection{Versión 0.1 --- 28 de febrero a 14 de marzo}
% Primera iteración

\subsubsection{Objetivos}
Diseñar y construir una aplicación como base para el proyecto, sin reutilizar código del prototipo pero con las mismas funcionalidades, e incluyendo documentación y pruebas.
Se realizarán modificaciones en la arquitectura general de la aplicación para facilitar la generación de ejercicios.

Se propone la idea de implementar un generador de ejercicios aleatorios basado en un algoritmo genético.

\subsubsection{Desarrollo}

\subsubsection{Resultados}
Implementación completa y estable del procesador de expresiones regulares y del algoritmo de Aho-Sethi-Ullman.

Interfaz funcional y preparada para ser ampliada con elementos posteriores.

Se añade funcionalidad de generación de problemas de tipo Aho-Sethi-Ullman implementando un algoritmo de búsqueda aleatoria.
La generación se realiza en función a una serie de parámetros dados por el usuario.

\subsection{Versión 0.2 --- 14 de marzo a 28 de marzo}
% Segunda iteración

\subsubsection{Objetivos}
Implementación del algoritmo de Thompson para la resolución de problemas de expresiones regulares.

\subsubsection{Desarrollo}

\subsubsection{Resultados}

El algoritmo de Thompson requeriría capacidades de dibujado de grafos para ser implementado correctamente.
Se implementa un tipo de problema relacionado, el de construcción de subconjuntos, a partir del cual implementaremos el de Thompson más adelante.

Permite asimismo la generación de problemas de construcción de subconjuntos mediante un algoritmo de búsqueda aleatoria.
Se comprueba que este algoritmo no resulta eficiente para este tipo de problemas, y se plantea la implementación de un algoritmo genético.
Completa la generación de documentos latex y en formato Moodle XML, incluyendo opciones de respuesta generadas por el propio programa.

\subsection{Versión 0.3 --- 28 de marzo a 11 de abril}
% Tercera iteración

\subsubsection{Objetivos}
Se plantean mejoras sobre el contenido de los ejercicios de Aho-Sethi-Ullman y construcción de subconjuntos.

Asimismo, se plantea el uso de algoritmos genéticos para la generación de problemas de construcción de subconjuntos.

\subsubsection{Desarrollo}

\subsubsection{Resultados}
Se implementan los cambios solicitados para los ejercicios.

Añadidas las bases para la implementación de algoritmos genéticos, como la posibilidad de realizar operaciones de mutación sobre las expresiones regulares.

Se prepara la implementación de generación concurrente de problemas, sin completar la funcionalidad.

\subsection{Versión 0.4 --- 11 de abril a 25 de abril}
% Cuarta iteración

\subsubsection{Objetivos}

\subsubsection{Desarrollo}

\subsubsection{Resultados}

\subsection{Versión 0.5 --- 25 de abril a 9 de mayo}
% Quinta iteración

\subsubsection{Objetivos}

\subsubsection{Desarrollo}

\subsubsection{Resultados}

\subsection{Versión 0.6 --- 9 de mayo a 23 de mayo}
% Sexta iteración

\subsubsection{Objetivos}

\subsubsection{Desarrollo}

\subsubsection{Resultados}


\bibliographystyle{plain}
\bibliography{PLQuiz}

\end{document}