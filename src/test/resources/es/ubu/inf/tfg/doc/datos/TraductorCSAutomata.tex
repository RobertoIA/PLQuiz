

\paragraph{{numero}.-}\label{p{numero}}
Completa la tabla de función de transición para el AFD que se obtendría de aplicar el método de construcción de subconjuntos al AFND de la figura
    \myincludegraphics{}
\rowcolors{2}{gray!25}{white}
\begin{tabular} {c@{\hspace{4mm}}ccc@{\hspace{4mm}}l}
\toprule % =============================
$\mathcal{Q}$ & a & b & c & \emph{NFA states}\\
\midrule %-------------------------------
\h{(A)} & \h{B} & \h{C} & \h{D} & \h{0-2, 4, 5, 7-9, 11, 13}\\
\h{B} & \h{E} & \h{F} & \h{D} & \h{3, 8-11}\\
\h{C} & \h{E} & \h{C} & \h{D} & \h{5-9, 11}\\
\h{(D)} & \h{B} & \h{C} & \h{D} & \h{1, 2, 4, 5, 7-9, 11-13}\\
\h{E} & \h{E} & \h{F} & \h{D} & \h{9-11}\\
\h{F} & \h{F} & \h{F} & \h{F} & \h{$\varnothing$}\\
\bottomrule % =============================
\end{tabular}
